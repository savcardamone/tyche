\section{Quantum Monte Carlo}
Consider the Schr\"{o}dinger equation for a molecular system whose $3n$ electronic
and $3N$ nuclear degrees of freedom will be denoted collectively by the position
vectors $\vec{r}$ and $\vec{R}$, respectively,
%
\begin{equation}\label{eq::ElectronicSchrodinger}
  \elecham\elecwfn = E\elecwfn \,,
\end{equation}
%
where $\elecwfn$ is the (unknown) electronic wavefunction and $\elecham$ is the
electronic Hamiltonian operator, whose functional form can be written
%
\begin{align}\label{eq::ElectronicHamiltonian}
  \elecham &=
  - \frac{\hbar^2}{2m}\nabla^2_{\vec{r}}
  - \sum_{I=0}^N\sum_{i=0}^n \frac{Z_I}{|\vec{R}_I - \vec{r}_i|}
  + \sum_{i<j}^n \frac{1}{|\vec{r}_i - \vec{r}_j|}
  + \sum_{I<J}^N \frac{Z_IZ_J}{|\vec{R}_I - \vec{R}_J|} \nonumber \\
  &= \telec + \vne + \vee + \vnn \,,
\end{align}
%
where $\telec, \vne, \vee, \vnn$ correspond to the electronic kinetic, electron-
nuclear attractive potential, electron-electron repulsive potential and nuclear-
nuclear repulsive potential, respectively.
We draw the reader's attention to the parametric dependence on the nuclear
degrees of freedom, which results from application of the Born-Oppenheimer
approximation (the electrons adjust instantaneously to any nuclear motion, and
consequently the nuclei can be thought of as static). The application of the
Born-Oppenheimer approximation also permits our omission of the nuclear kinetic
energy from \eqref{eq::ElectronicHamiltonian}.

\section{Variational Monte Carlo}

\section{Diffusion Monte Carlo}
The time-dependent Schr\"{o}dinger equation (TDSE) for a molecular system is written
%
\begin{equation}\label{eq::TDSE}
  \im\hbar\ddt{t}\tdelecwfn{t} = \elecham\tdelecwfn{t} \,,
\end{equation}
%
where $\tdelecwfn{t}$ is the time-dependent electronic wavefunction. General solutions
to the TDSE can be written as the product of a complex exponential and the starting
condition $\tdelecwfn{t_0}$
%
\begin{equation}
  \tdelecwfn{t} = \tdelecwfn{t_0}\exp[-\im E t / \hbar] \,,
\end{equation}
%
where $E$ is the energy of $\tdelecwfn{t_0}$. Given Euler's identity, it is clear that
this solution is oscillatory in time. Beginning with the TDSE may seem
an odd starting point given we have no intentions of simulating the dynamics of
the molecular system characterised by $\tdelecwfn{t}$, but we'll find that it allows
us to evolve $\tdelecwfn{t}$ to the ground state of $\elecham$ through the
dynamics codified in \eqref{eq::TDSE}.

Our first step is to perform a Wick rotation, making the substitution $\tau = \im t$
(to ``imaginary time'') in \eqref{eq::TDSE}, yielding
%
\begin{equation}\label{eq::WickTDSE}
  -\hbar\ddt{\tau}\tdelecwfn{\tau} = \elecham\tdelecwfn{\tau} \,.
\end{equation}
%
This seemingly innocuous operation results in solutions to \eqref{eq::WickTDSE} having
decaying exponentials in imaginary time, rather than complex exponentials in ``real''
time. This gives rise to a particaularly useful property: in the infinite imaginary
time limit (i.e. as $\tau \rightarrow \infty$), contributions to $\tdelecwfn{\tau}$
from all but the lowest energy eigenfunctions of $\elecham$ tend to zero. So evolving
\eqref{eq::WickTDSE} in imaginary time rather magically leaves us with the ground state
solution of $\elecham$!
