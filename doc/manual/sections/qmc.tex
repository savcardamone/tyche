\section{Quantum Monte Carlo}
Consider the Schr\"{o}dinger equation for a molecular system whose $3n$ electronic
and $3N$ nuclear degrees of freedom will be denoted collectively by the position
vectors $\vec{r}$ and $\vec{R}$, respectively,
%
\begin{equation}
  \elecham\elecwfn = E\elecwfn \,,
\end{equation}
%
where $\elecwfn$ is the (unknown) electronic wavefunction and $\elecham$ is the
electronic Hamiltonian operator, whose functional form can be written
%
\begin{equation}\label{eq::ElectronicHamiltonian}
  \elecham =
  -\frac{\hbar^2}{2m}\nabla^2_\vec{r}
  - \sum_{I=0}^N\sum_{i=0}^n \frac{Z_I}{|\vec{R}_I - \vec{r}_i|}
  + \sum_{i<j}^n \frac{1}{|\vec{r}_i - \vec{r}_j|}
  + \sum_{I<J}^N \frac{Z_IZ_J}{|\vec{R}_I - \vec{R}_J|} \,,
\end{equation}
%
We draw the reader's attention to the parametric dependence on the nuclear
degrees of freedom, which results from application of the Born-Oppenheimer
approximation (the electrons adjust instantaneously to any nuclear motion, and
consequently the nuclei can be thought of as static). The application of the
Born-Oppenheimer approximation also permits our omission of the nuclear kinetic
energy from \eqref{eq::ElectronicHamiltonian}.

\section{Variational Monte Carlo}
